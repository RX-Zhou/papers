\documentclass[12pt]{nature}
\usepackage{graphicx,color,multirow,bm,braket,amsmath,soul,epstopdf,lineno}
\renewcommand{\figurename}{Figure}
\def\bk{{\bf k}}
\def\bq{{\bf q}}
\def\ve{\varepsilon}
\def\e{\epsilon}
\setlength{\textheight}{23cm}
\setlength{\voffset}{-0.5cm}
\setlength{\parindent}{0cm}
\setlength{\parskip}{10pt}
\spacing{1.5}
 \def\@maketitle{%
   \newpage\spacing{1.1}\setlength{\parskip}{10pt}%
     {\Large\bfseries\noindent\sloppy \textsf{\@title} \par}%
     {\noindent\sloppy \@author}%
 \vspace{-0.6cm} }    
\renewcommand\refname{\vspace{-38pt}\setlength{\parskip}{8pt}}
 
\begin{document}

\title{Origin of the crossover from polarons to Fermi liquids in transition metal oxides}
\author{Carla Verdi, Fabio Caruso, Feliciano Giustino}

\maketitle 

\vspace*{-0.2cm} 
\begin{affiliations} 
\item[] Department of Materials, University of Oxford, Parks Road, Oxford OX1 3PH, United Kingdom.
\\ \small \normalfont Correspondence and requests for materials should be addressed to 
F.G. (email: feliciano.giustino@materials.ox.ac.uk).
\end{affiliations} 
 
%\linenumbers
 
\begin{abstract} 
Transition metal oxides host a wealth of exotic phenomena ranging from charge, 
orbital, and magnetic order to nontrivial topological phases and superconductivity. In order to translate 
these unique materials properties into device functionalities these materials 
must be doped, however the nature of carriers and their conduction mechanism at the 
atomic scale remain unclear. Recent angle-resolved photoelectron spectroscopy investigations provided 
insight into these questions, revealing that the carriers of prototypical metal oxides undergo a 
transition from a polaronic liquid to a Fermi liquid regime with increasing doping. 
Here, by performing \textit{ab initio} many-body calculations of angle-resolved 
photoemission spectra of titanium dioxide, we show that this transition originates from 
non-adiabatic polar electron-phonon coupling, and occurs when the frequency of plasma oscillations 
exceeds that of longitudinal-optical phonons. This finding suggests that a universal mechanism may 
underlie polaron formation in transition metal oxides, and provides a pathway 
for engineering emergent properties in quantum matter.
\end{abstract}
\vspace{0.3cm} 

Elucidating the nature of charge carriers in doped transition metal oxides (TMOs) is key 
to understanding the mechanism of electrical conduction in these multifunctional materials. In 
conducting oxides the infrared-active vibrations can couple strongly to electrons, leading to the 
formation of polarons\cite{Devreese2009}. Polarons are electrons dressed by a phonon cloud\cite{Mahan}, 
and represent a paradigmatic example of emergent state in condensed matter.
Depending on their mass and size, polarons exhibit widely different conduction 
mechanisms, from band-like transport to thermally-activated hopping transport\cite{Mott1969, Ziman}. 
Despite being central to the science and technology of oxides, little is known about the 
properties of polaronic states.

The interest in electron-phonon coupling and polaronic quasiparticles in TMOs has been reinvigorated 
by recent angle-resolved photoelectron spectroscopy (ARPES) experiments\cite{Moser2013,
Chen2015,Cancellieri2016,Baumberger2016,Yukawa2016}. 
The signature of polaronic behavior in ARPES spectra is the appearance of satellites below the 
conduction band, at integer multiples of the optical phonon energy. This is reported 
in Fig.~\ref{fig1}a-b for the paradigmatic case doped anatase TiO$_2$\cite{Moser2013}. These 
pioneering measurements showed that, by increasing the carrier concentration, polaronic satellites 
gradually evolve into the photoemission kinks observed in metals and superconductors\cite{Damascelli2003}, 
see Fig.~\ref{fig1}c. It was proposed that this crossover reflects the evolution of charge carriers from 
polarons to a Fermi liquid\cite{Moser2013,Baumberger2016}. In order to clarify the origin of this transition 
without making any {\it a priori} assumption about the underlying mechanism, first principles 
calculations are urgently called for. However, the investigation of polaronic features 
in ARPES spectra from first principles and their evolution with doping is exceptionally challenging 
and has never been reported before.

In the following we focus on the prototypical example of anatase TiO$_2$. On top of its well-known 
applications in solar energy harvesting\cite{Hardin2012, Snaith2014} and superhydrophilic 
technology\cite{Fujishima1972, Fujishima2000}, this material is also being investigated in the quest 
for transparent conducting oxides based on non-toxic and Earth-abundant elements\cite{Furubayashi2005,
Ellmer2012}. Despite its pivotal role in a broad range of technologies, the nature of the charge carriers 
in anatase is still controversial\cite{DeAngelis2014}. Here we address these issues by calculating ARPES 
spectra and polaron wavefunctions entirely from first principles. We develop a novel theoretical and 
computational framework that allows us to investigate polarons and Fermi liquid quasiparticles on the 
same footing, and without resorting to any empirical parameters. Using this new approach, we show how 
the interplay between the dynamical screening of the electron plasma and the Fr\"ohlich electron-phonon 
coupling is responsible for the transition between polaronic and Fermi liquid states. We propose that 
the mechanism identified in this work may be universal, and also applies to other oxides such as 
SrTiO$_3$ and ZnO. 


\section*{\textsf{Results}} 

\section*{\textsf{\small Angle-resolved photoemission spectra}}

Our calculated ARPES spectra are shown in Fig.~\ref{fig1}\mbox{d-f}, for the same doping levels as in the 
measurements of ref.~\citenum{Moser2013}, reproduced in Fig.~\ref{fig1}a-c. These maps show the bottom 
of the conduction band of \mbox{$n$-doped} anatase TiO$_2$, for three doping levels in the range 
$10^{18}$ to $10^{20}$~cm$^{-3}$. All the spectra exhibit a bright parabolic band, whose
size increases with doping. This reflects the rise of the Fermi energy inside the conduction band 
as the electron density increases. Besides this bright feature, panels a-b (experiments) and d-e
(calculations) show each a pair of satellites, a bright one at a binding energy around 0.1~eV, and
a dim one near 0.25~eV. These features are identified as polaronic effects\cite{Moser2013}. Moving on 
to higher doping in panels c and f, the satellites disappear and are replaced by band structure kinks
near 0.1~eV. Overall, our calculated ARPES spectra are in remarkable agreement with the experiments 
of ref.~\citenum{Moser2013}. In order to achieve this unprecedented level of precision without any 
adjustable parameters, we developed an innovative computational framework.

In our calculations the photoelectron intensity maps are obtained using the single-particle spectral 
function $A(\bk,\omega)$, where $\hbar\bk$ and $\hbar\omega$ are the electron momentum and binding
energy, respectively, and $\hbar$ is the reduced Planck constant. The spectral function is calculated 
using the state-of-the-art cumulant expansion, which has been employed recently to successfully 
describe plasmon satellites\cite{Aryasetiawan1996,Guzzo2011,Lischner2013,Caruso2015}. 
The cumulant expansion can naturally be applied to investigate the spectral properties of a polaronic 
system since the theory stems from the exact solution of the Fr\"ohlich electron-boson coupling 
Hamiltonian\cite{Langreth1970,Rehr2014}. In this formalism the spectral function is expressed 
as\cite{Hedin1980}:
 \begin{equation} \label{eq.spec.cum}
  A(\bk,\omega) = \frac{1}{2\pi}\,\sum_n {\rm Re}\! \int_{-\infty}^{+\infty}\! \!\mathrm{d}t\,
  e^{i(\omega-\ve_{n\bk}/\hbar) t}\,e^{C_{n\bk}(t)},
 \end{equation}
where $n$ is the electron band index, $\ve_{n\bk}$ is the electron eigenvalue in absence of many-body 
interactions, and $t$ is the time variable. The quantity $C_{n\bk}$ is the so-called cumulant
function, and can be calculated by using the standard electron-phonon self-energy $\varSigma_{n\bk}$ as a seed
(see Methods and Supplementary Note~1):
 \begin{equation} \label{Sigma} 
  \varSigma_{n\bk}(\omega)=\sum_{m\nu}\int \!\frac{\mathrm{d}\bq}{\varOmega_{\rm BZ}} 
  |g_{mn\nu}(\bk,\bq)|^2 \times 
  \left[\frac{n_{\bq\nu}+f_{m\bk+\bq}}{\omega-\ve_{m\bk+\bq}
  +\omega_{\bq\nu}-i\eta} +\frac{n_{\bq\nu}+1-f_{m\bk+\bq}}{\omega-\ve_{m\bk+\bq}\!-
  \omega_{\bq\nu}-i\eta} \right] .
 \end{equation}
Here $g_{mn\nu}(\bk,\bq)$ is the electron-phonon vertex, and describes the probability amplitude for 
an electron in the initial state $|n\bk \rangle$ to be scattered into the final state $|{m\bk+\bq}\rangle$ 
by a phonon with momentum $\hbar\bq$ and energy $\hbar\omega_{\bq\nu}$ in the branch $\nu\,$\cite{GiustinoRMP}. 
The terms $n_{\bq\nu}$ and $f_{m\bk+\bq}$ denote the Bose-Einstein and 
Fermi-Dirac occupations, respectively; $\varOmega_{\rm BZ}$ is the Brillouin zone volume, and $\eta$ 
a positive infinitesimal. The numerical evaluation of Eqs.~\eqref{eq.spec.cum}-\eqref{Sigma} is very 
challenging owing to the singular nature of the Fr\"ohlich interaction at long wavelengths\cite{Frohlich1954}. 
To overcome this challenge we use the Wannier function technique of refs.~\citenum{EPW2007,Verdi2015}, 
as implemented in the EPW code\cite{EPW2016}. 

The strength of the electron-phonon interaction is most commonly expressed in terms of a single 
parameter $\lambda$, which describes the enhancement of the electron mass from the band effective 
mass $m_{\rm b}$ to the polaron mass $m^*$ via $m^*=m_{\rm b}(1+\lambda)$, as well as the renormalization 
of the Fermi velocity\cite{Grimvall}. To determine this parameter we first extract the band structures 
underlying \mbox{Fig.~\ref{fig1}d-f}, by tracking the maxima in the energy distribution curves. This 
procedure yields a set of parabolic bands and their satellites in \mbox{Fig.~\ref{fig1}g-h}, and a 
distorted parabola in \mbox{Fig.~\ref{fig1}i} (blue curves). For comparison we also report the electronic 
bands in absence of electron-phonon interactions (red curves). For carrier concentrations of $5\times10^{18}$ 
and $3\times10^{19}$~cm$^{-3}$, polaron satellites are clearly visible in \mbox{Fig.~\ref{fig1}g-h}, 
whereas at $3.5\times10^{20}$~cm$^{-3}$ we see a band structure kink but no satellites. From these band 
structures we obtain the mass enhancement parameter as the ratio between the Fermi velocity of the bare band, 
$v^0_{\rm F}$, and that of the dressed band, $v_{\rm F}$, as indicated in Fig.~\ref{fig1}i: 
$\lambda_{\bf k}=v^0_{\bf k\rm F}/v_{\bf k\rm F} - 1$~\cite{Grimvall}, where we explicitly included the 
dependence on the wavevector at the Fermi surface. As we move from the lowest to the highest doping level 
we obtain $\lambda_{\bf k}=0.73$, 0.70, and 0.20, respectively. Our calculated value at intermediate doping 
is in excellent agreement with the mass enhancement determined in experiments, $\lambda=0.7$~\cite{Moser2013}. 

In Fig.~\ref{fig1}k we show the spectrum calculated at intermediate doping for electron momenta 
along $\Gamma$X and $\Gamma$Z, as well as the Fermi surface cut. Here we see that the Fermi 
surface pocket is elongated along the $\Gamma$Z direction. This elongation reflects the anisotropic 
character of the band effective masses, which we calculate to be $m_{\rm b}^\perp =0.40\,m_{\rm e}$ and 
$m_{\rm b}^\parallel =4.03\,m_{\rm e}$ along $\Gamma$X and $\Gamma$Z, respectively ($m_{\rm e}$ is the 
electron mass). Surprisingly this anisotropy is not reflected in the electron-phonon coupling strength: 
our calculations indicate that the mass enhancement $\lambda_{\bf k}$ varies by less than 10\% along 
the [100], [110], and [001] directions, leading to an average value of $\lambda=0.73$, 0.70 and 0.19 
for the three doping levels considered (see also Supplementary Note~2). These results are in good 
agreement with resonant inelastic X-ray scattering experiments on anatase TiO$_2$\cite{Moser2015}. 

\section*{\textsf{\small Origin of satellites and kinks}}

Having established that our calculations can accurately reproduce experimental spectra without 
adjustable parameters, we now proceed to identify the mechanisms that drive the formation of 
polaron satellites by selectively turning off individual components of the calculations. 
The energy separations between the quasiparticle bands and the first satellite in Fig.~\ref{fig1}d 
and Fig.~\ref{fig1}e are 106~meV and 124~meV, respectively. In the case of Fig.~\ref{fig1}f 
the kink appears at a binding energy of 100~meV. These energy scales are compatible with a 
Fr\"ohlich-type coupling to the longitudinal-optical (LO) $E_u$ phonon at 109~meV (see phonon 
dispersions in Supplementary Fig.~1). This vibrational mode corresponds to the stretching of 
the Ti-O bonds in the $ab$ plane, as shown in Fig.~\ref{fig2}e. Another candidate bosonic mode is 
the $c$-axis $A_{2u}$ phonon at 88~meV\cite{Moser2015}, however the energy of this mode appears too 
small to account for the satellites and kinks in Fig.~\ref{fig1}. In order to quantify the importance 
of the $E_u$ phonon, in Fig.\ref{fig2}a we compare two calculations: the complete spectrum at intermediate 
doping and a calculation where the coupling to modes with energy above 100~meV is 
artificially suppressed. We see that, upon removing high-energy phonons, the 
intensity of the first satellite decreases and the second satellite disappears. The effective mass is also 
visibly lower, in fact the analysis of the mass enhancement parameter yields $\lambda=0.3$, to be compared 
to the total coupling $\lambda=0.7$. It follows that the $E_u$ phonon contributes $\sim60\%$ of the total 
coupling, hence it represents the primary mechanism behind the satellites.

In Fig~\ref{fig2}b we test the importance of many-body correlations beyond the one-shot Migdal 
approximation. The spectrum at intermediate doping, obtained from the cumulant expansion, 
is shown together with the result of a calculation within the one-shot Migdal approximation 
(see Methods). The Migdal approximation is obtained from the complete electron-phonon self-energy by 
neglecting the three-point vertex\cite{Mahan}. In this approximation the electron-phonon self-energy 
contains only non-crossing diagrams. The additional approximation adopted in first principles 
calculations is to replace the fully renormalized  electron and phonon propagators by those evaluated 
within density functional theory\cite{GiustinoRMP}. We name this choice \textit{one-shot} Migdal
approximation to emphasize the lack of self-consistency in the electron propagator. From Fig.~\ref{fig2}b 
we notice that the one-shot Migdal approximation fails twice: firstly, the separation between the 
quasiparticle band and the first satellite is too large as compared to experiment (151~meV instead 
of $\sim\:$100~meV); secondly, the dim satellite around 0.25~eV is completely missing. This test highlights 
the crucial role of high-order electron-phonon correlations in the description of polarons in conducting 
oxides. It would be interesting to see whether the deficiencies of the one-shot Migdal approximation 
could be avoided by performing a fully self-consistent Migdal calculation,\cite{Marsiglio1990} and to 
establish the importance of crossing diagrams which are included in the cumulant expansion; however 
this test is currently out of reach.

We now move to discuss the origin of the crossover between satellites and kinks in the spectra as a 
function of doping. The Fr\"ohlich theory of polarons\cite{Frohlich1954} considers a single electron 
added to a polar insulator. This description does not take into account that, as more electrons are 
added to the system, the polar electron-phonon coupling is weakened by the electronic screening of the 
charge carriers. Following ref.~\citenum{Mahan}, we treat this issue by screening the electron-phonon 
vertex by the frequency-dependent Lindhard dielectric function with the calculated effective mass and 
dielectric permittivity of anatase TiO$_2$ (see also Methods). For completeness we show in Supplementary 
Fig.~2 the dielectric screening as a function of carrier density and how the Fr\"ohlich electron-phonon 
vertex is influenced by doping. In order to illustrate the importance of carrier screening, in 
Fig.~\ref{fig2}c we compare two scenarios: the spectrum calculated at high doping by accounting for 
electronic screening and the same system, but this time ignoring the screening 
of the electron-phonon coupling by doped carriers. 
The result is striking: in the absence of screening one obtains a 
sharp polaron satellite, in stark disagreement with experimental evidence. This comparison indicates 
that a correct description of the electronic screening is absolutely crucial to capture the photoemission 
kink at high doping. On the contrary, when we repeat the comparison of Fig.~\ref{fig2}c for the cases of 
low and intermediate doping, we find a completely different picture: at these doping levels the electronic 
screening does not play any significant role. 

These conflicting observations can be rationalized by inspecting the timescales of lattice vibrations 
and electronic screening. The $E_u$ phonon vibrates with a period $T_{\rm ph}=38$~fs. The characteristic 
response time of the electronic screening is set by the plasma frequency of doped carriers, $T_{\rm el} = 
2\pi/\omega_{\rm p}$. In the case of $n$-doped TiO$_2$ the electrons occupy a singly-degenerate parabolic 
band minimum, therefore $\omega_{\rm p}= (n e^2/\epsilon_0 \epsilon_\infty m_{\rm b})^{1/2}$, 
where $n$ is the electron density, $\epsilon_0$ the permittivity of vacuum, and 
$\epsilon_\infty$ the high-frequency dielectric constant of TiO$_2$\cite{Mahan}. For the electron 
densities considered in Fig.~\ref{fig1}a-c we calculate $T_{\rm el} = 122$~fs, 50~fs, and 15~fs, 
respectively. From these values we deduce that, at the lowest doping, the carriers are too slow to 
screen the long-range electric field generated by the oscillation of the $E_u$ phonon (122~fs vs.\ 
38~fs). In this case the screening is ineffective and the Fr\"ohlich interaction dominates the spectrum. 
On the contrary, at the highest doping the electrons oscillate faster than the LO phonon (15~fs vs.\ 
38~fs). In this case the electronic screening is almost complete and the Fr\"ohlich coupling is largely 
suppressed. In this regime the strength of the kink depends critically on the carrier concentration; 
with increasing doping the coupling to the LO phonons is gradually suppressed, and the ARPES spectrum is 
dominated by the weaker coupling of carriers to non-polar phonons\cite{Baumberger2016}. These considerations 
are summarized in Fig.~\ref{fig2}d, where we compare the energy of the $E_u$ phonon with the plasma energy 
$\hbar\omega_{\rm p}$ of the carriers, and we monitor the evolution of the coupling strength with doping. 
We can identify two regions: a  polaronic regime, corresponding to the situation $\omega_{\rm p}<
\omega_{\rm ph}$, and a  Fermi liquid regime, corresponding to $\omega_{\rm p} > \omega_{\rm ph}$. 
In the polaronic region the electronic screening is ineffective, we see satellites in the spectra, and 
the electron-phonon mass enhancement is not sensitive to the doping level. In the Fermi liquid region 
the Fr\"ohlich coupling is strongly suppressed, polaron satellites are gradually replaced by photoemission 
kinks, and the coupling strength decreases. To further validate this trend, we performed additional 
calculations for a doping concentration of $1\times10^{20}$~cm$^{-3}$ in the transition region. We 
found that one satellite is still present in the spectrum (Supplementary Fig.~3). A careful investigation 
of the mass renormalization parameter yields \mbox{$\lambda=0.34$}, thus confirming that the electron-phonon 
coupling is weakened by the electronic screening. The present analysis reveals that the origin of the 
crossover from polarons to a Fermi liquid in the ARPES spectra of doped TiO$_2$ is to be found in a 
novel form of electron-phonon coupling, which we refer to as a non-adiabatic Fr\"ohlich interaction.

Given the qualitative change in the band structures at the crossover from the polaronic to the Fermi 
liquid regime, it is natural to ask how this evolution is reflected in the wavefunctions. In order to 
explore this aspect we calculate the polaron wavefunctions by generalizing the perturbation theory of 
ref.~\citenum{Mahan} to {\it ab initio} calculations. 
Fig.~\ref{fig2}f shows the square moduli of the polaron wavefunctions at the bottom of the conduction 
band near the maximum of the polaron wavefunction and a few unit cells away: in blue 
the intermediate doping level, which is inside the polaronic region of Fig.~\ref{fig2}d; 
in yellow the high doping level, in the Fermi liquid region. The corresponding envelope 
functions are also shown. Here we see that the 
wavefunction of an electron in the Fermi liquid region is essentially akin to a periodic Bloch function. 
On the contrary, the wavefunction of an electron in the polaronic region shows spatial localization. 
To quantify the size of the polaron in this latter case we define a polaron radius $r_{\rm p}$ using 
the half width at half maximum. From Fig.~\ref{fig2}f we obtain $r_{\rm p}=5.7$~nm. This result indicates 
that, despite the qualitative changes in the ARPES spectra, we are in the presence of large polarons 
throughout the entire doping range, and that polarons in TiO$_2$ are considerably more delocalized than
previously thought\cite{Moser2013}. Since thermally-activated hopping transport corresponds 
to $r_{\rm p}$ of the order of the lattice constant\cite{Emin}, our finding supports the 
notion that electrical conduction in anatase TiO$_2$ takes place via standard band-like transport. 
To avoid ambiguities we emphasize that the present result refers to intrinsic
mobile polarons, not to localized electronic defect states such as those associated with O vacancies
and which are not mobile\cite{Setvin2014}.

\vspace{10pt}

\section*{\textsf{Discussion}}

The non-adiabatic Fr\"ohlich mechanism identified here is simple enough that it is likely to play
a role in many other conducting oxides. In order to test this hypothesis we estimate the critical 
density for the crossover in doped SrTiO$_3$\cite{Baumberger2016} and doped ZnO\cite{Yukawa2016}. 
In SrTiO$_3$ the plasma energy for an electron concentration of $10^{20}$~cm$^{-3}$ is 
$\hbar\omega_{\rm p}\sim64$~meV\cite{Gervais1993}, therefore to match the 100~meV LO phonon of 
SrTiO$_3$ one would need a doping level of $2.6\times10^{20}$~cm$^{-3}$. In the experiments 
of ref.~\citenum{Baumberger2016} the carriers are confined in a thin surface layer corresponding 
to approximately 3~unit cells, therefore we estimate the critical carrier density for the two-dimensional 
electron liquid in the range $3\times10^{13}$~cm$^{-2}$. This value is remarkably close to the 
critical density determined experimentally, $4\times10^{13}$~cm$^{-2}$~\cite{Baumberger2016}. 
More accurate calculations will need to take into account the two-dimensional screening at the 
surface of SrTiO$_3$ and its quantum paraelectric nature. 
A similar analysis can be carried out for ZnO\cite{Yukawa2016}. In this case the energy of the 
highest LO phonon is 72~meV, while the plasma energy for a surface electron concentration of 
$7.5\times10^{13}$~cm$^{-2}$ is 320~meV\cite{Goldstein1982}. The transition would then appear 
around $3.8\times10^{12}$~cm$^{-2}$, slightly below the density reported in ref.~\citenum{Yukawa2016}. 
In agreement with our estimate, the spectra of ref.~\citenum{Yukawa2016} exhibit a behavior 
which is intermediate between kinks and satellites. 

In sum, our findings indicate that the electron-phonon coupling in TiO$_2$ is more complex than previously 
thought, and that the non-adiabatic Fr\"ohlich coupling could be the unifying mechanism behind the 
transition from polarons to Fermi liquids in conducting oxides; explicit {\it ab initio} calculations 
will be required to confirm this point. Looking further ahead, our work suggests that a fine control 
of the interplay between lattice vibrations and plasma oscillations may offer a 
pathway for investigating novel emergent states in quantum materials, and provide 
opportunities in the development of quantum technologies based on TMOs.

\vspace{10pt}

\section*{\textsf{Methods}} 

\section*{\textsf{\small Ground-state calculations}} \vspace*{-10pt}

{\it Ab initio} calculations were carried out for anatase TiO$_2$ (space group $I4_1/amd$), using
the experimental lattice parameter $a=3.784$~\AA\cite{Horn1972}. We used density-functional theory
(DFT) within the generalized gradient approximation of Perdew, Burke, and Ernzerhof\cite{PBE1996}.
The core-valence interaction was described by means of norm-conserving pseudopotentials, with the
semicore Ti-$3s$ and Ti-$3p$ states explicitly taken into account. Electron wavefunctions were
expanded in a planewaves basis set with kinetic energy cutoff of 200~Ry, and the Brillouin zone
was sampled using a $6\times6\times6$ Monkhorst-Pack mesh. Lattice-dynamical properties were
calculated using  density functional perturbation theory (DFPT). All DFT and DFPT calculations 
were performed using the Quantum ESPRESSO\cite{QuantumEspresso} package.

\section*{\textsf{\small Electron-phonon coupling}} \vspace*{-10pt}

Calculations of electron-phonon couplings were performed using the EPW code\cite{EPW2016},
the cumulant expansion was performed separately (see Supplementary Note~1). The Fr\"ohlich electron-phonon 
matrix element was calculated using the method of ref.~\citenum{Verdi2015}. An accurate description of 
the Fr\"ohlich vertex as described in ref.~\citenum{Verdi2015} is essential: ignoring this effect leads 
to a severe underestimation of the mass enhancement\cite{Zhukov2014}. To evaluate Eq.~\eqref{Sigma} 
we computed electronic and vibrational states as well as the scattering matrix elements on a 
$4\times4\times4$ Brillouin-zone grid. These quantities were interpolated with \textit{ab initio} 
accuracy onto a fine grid with $2\cdot10^6$ random $\bq$-points using EPW. The positive 
infinitesimal $\eta$ was set to 10~meV. Temperature effects were accounted for by including the 
Fermi-Dirac occupation in the spectral function, corresponding to the experimental temperature of 20~K.

\section*{\textsf{\small Doping}} \vspace*{-10pt}

Doping was included using the rigid-band approximation since the system is degenerate. 
In fact, by considering the Mott criterion for the metal-insulator transition\cite{Mott1990}, 
$a_0^\ast\,n_c^{1/3}\approx0.25$ where $a_0^\ast=\hbar^2\epsilon_0/(e^2 m_{\rm b})$ is the effective 
Bohr radius, we obtain a value $n_c=1.3\times10^{18}$~cm$^{-3}$ for the critical density that is below 
the doping levels investigated. The screening of the electron-phonon interaction arising from the 
doped carriers was taken into account by computing the dielectric function $\epsilon(\bq,\omega)$ in 
the random-phase approximation\cite{Mahan, Hedin1965}, for a homogeneous electron gas with the calculated 
effective mass $m_b$ and dielectric permittivity $\epsilon_\infty$ of anatase TiO$_2$\cite{Mahan, Hedin1965} 
(see Supplementary Note~3 for a rationale). In this expression the plasma frequency directly reflects 
the carrier density, therefore the influence of doping on the dielectric screening in anatase 
TiO$_2$\cite{Mardare2004} is included in the calculations. The resulting non-adiabatic matrix element 
is $g_{mn\nu}^{\rm NA}(\bk,\bq)= g_{mn\nu}(\bk,\bq)/\epsilon(\bq,\omega_{\bq\nu}+i/\tau_{n\bk})$, where 
$\tau_{n\bk}$ is the electron lifetime near the band edge. Here we use $\hbar/\tau_{n\bk} = 55$~meV 
from ref.~\citenum{Moser2013}. We checked that this choice does not affect our conclusions (Supplementary 
Fig.~4). We note that, owing to the strong anisotropy of the Fermi surface in \mbox{$n$-doped} anatase 
TiO$_2$, it is possible that the nominal doping levels estimated in ref.~\citenum{Moser2013} using the 
Fermi momentum may underestimate the actual carrier densities in the measured samples.

\section*{\textsf{\small Angle-resolved photoemission spectra}} \vspace*{-10pt}

The cumulant function used in the calculation of the ARPES spectra is obtained from the self-energy of 
Eq.~\eqref{Sigma} as follows\cite{Aryasetiawan1996}:
$ C_{n\bk}(t)=(\pi\hbar)^{-1}\int_{-\infty}^{\infty}{\rm d}\omega \,
 \,\theta(\omega-\ve_{n\bk}/\hbar)\, e^{(i\omega+\eta)t}\,
 {\rm Im}\,\varSigma_{n\bk}(\ve_{n\bk}/\hbar-\omega)/(\omega-i\eta)^2
$, where $\theta$ is the Heaviside function. More details on the cumulant expansion are provided in 
Supplementary Note~1. The calculations within the one-shot Migdal approximation shown in Fig.~\ref{fig2}b 
were performed by using directly Eq.~\eqref{Sigma} inside the spectral function\cite{GiustinoRMP}: 
$ A(\bk,\omega)= 2\pi^{-1}\sum_n |\,{\rm Im}\,(\,\hbar\omega-\epsilon_{n\bk}-\varSigma_{n\bk})^{-1}| $.
The spectra were computed without including photon matrix elements effects\cite{Damascelli2003}. 
This approximation is justified since the variation of the orbital character over the $\sim$0.1~eV 
binding energy range shown in Fig.~\ref{fig1} is negligible\cite{Damascelli2003}. 
In order to extract the bands shown in Fig.~\ref{fig1}g-i we proceeded as follows. For the 
quasiparticle bands, which are the brightest features in Fig.~\ref{fig1}d-f, we calculate the poles 
of the electron Green's function as $E_{n\bk}=\ve_{n\bk}+{\rm Re}\,\varSigma_{n\bk}(E_{n\bk})$, where 
$\ve_{n\bk}$ and $E_{n\bk}$ are the bare and the renormalized dispersions, respectively. For the 
satellites in Fig.~\ref{fig1}d-e the previous procedure is not applicable since these features are 
not poles of the Green's function. In this case we directly identify the maxima in the energy 
dispersion curves\cite{Damascelli2003}, and obtain the pairs of parabolic bands which are visible 
in Fig.~\ref{fig1}g-h at binding energies below 0.1~eV. 

\section*{\textsf{\small Polaron wavefunction}} \vspace*{-10pt}

In order to calculate the wavefunction of the polaron we generalized the perturbation theory approach 
of ref.~\citenum{Mahan} to the case of multiple electron bands and phonon branches, as well 
as {\it ab initio} electron-phonon matrix elements. We found
$ \tilde{\psi}_{n\bk}(\mathbf r;\{\bm\tau_{\kappa p}\})\!=\!C \psi_{n\bk}({\bf r}) \times\left[1\!+\!
 \sum_{\nu}\int\!\frac{{\rm d}\bq}{\varOmega_{\rm BZ}}\,|g^{\rm NA}_{nn\nu}(\bk,\bq)|^2\right.$ 
 $\left.e^{i\bq\cdot\mathbf r}/(\ve_{n\bk}\!-\!\ve_{n\bk+\bq}\!-\!\omega_{\bq\nu})^2 \right]\!\ket{0_{\rm p}}$, 
where $\psi_{n\bk}$ and $\tilde{\psi}_{n\bk}$ are the electron wavefunction without and with 
electron-phonon interactions, respectively. $\{\bm\tau_{\kappa p}\}$ indicate the nuclear coordinates, 
$\ket{0_{\rm p}}$ is the phonon ground state and $C$ is a normalization constant. The details of the 
derivation are given in Supplementary Note~4. The wavefunctions in Fig.~\ref{fig2}f were obtained by 
setting $n\bk$ to the conduction band edge; the envelope functions were calculated by retaining only 
the term inside the square brackets in the above expression. A random grid with $2\cdot10^6$ points 
was used for the Brillouin-zone integration, and a small imaginary part $\eta=10$~meV was added to 
the denominator.

\section*{\textsf{\small Data availability}} \vspace*{-10pt}
The data that support the findings of this study are available from the corresponding author upon request, 
as is the modified version of the EPW package to include screening and the custom code for the 
cumulant expansion.

%\nolinenumbers

\vspace{0.5cm}
\section*{\textsf{References}}
%\bibliographystyle{naturemag}
%\bibliography{biblio-tio2}
\begin{thebibliography}{10}
\expandafter\ifx\csname url\endcsname\relax
  \def\url#1{\texttt{#1}}\fi
\expandafter\ifx\csname urlprefix\endcsname\relax\def\urlprefix{URL }\fi
\providecommand{\bibinfo}[2]{#2}
\providecommand{\eprint}[2][]{\url{#2}}

\bibitem{Devreese2009}
\bibinfo{author}{Devreese, J.~T.} \& \bibinfo{author}{Alexandrov, A.~S.}
\newblock \bibinfo{title}{Fr\"{o}hlich polaron and bipolaron: recent
  developments}.
\newblock \emph{\bibinfo{journal}{Rep. Prog. Phys.}}
  \textbf{\bibinfo{volume}{72}}, \bibinfo{pages}{066501}
  (\bibinfo{year}{2009}).

\bibitem{Mahan}
\bibinfo{author}{Mahan, G.~D.}
\newblock \emph{\bibinfo{title}{Many-Particle Physics}}
  (\bibinfo{publisher}{Plenum}, \bibinfo{address}{New York},
  \bibinfo{year}{1981}).

\bibitem{Mott1969}
\bibinfo{author}{Austin, I.~G.} \& \bibinfo{author}{Mott, N.~F.}
\newblock \bibinfo{title}{Polarons in crystalline and non-crystalline
  materials}.
\newblock \emph{\bibinfo{journal}{Adv. Phys.}} \textbf{\bibinfo{volume}{18}},
  \bibinfo{pages}{41--102} (\bibinfo{year}{1969}).

\bibitem{Ziman}
\bibinfo{author}{Ziman, J.~M.}
\newblock \emph{\bibinfo{title}{Electrons and Phonons}}
  (\bibinfo{publisher}{Oxford University Press}, \bibinfo{year}{1960}).

\bibitem{Moser2013}
\bibinfo{author}{Moser, S.} \emph{et~al.}
\newblock \bibinfo{title}{Tunable polaronic conduction in anatase {TiO$_2$}}.
\newblock \emph{\bibinfo{journal}{Phys. Rev. Lett.}}
  \textbf{\bibinfo{volume}{110}}, \bibinfo{pages}{196403}
  (\bibinfo{year}{2013}).

\bibitem{Chen2015}
\bibinfo{author}{Chen, C.}, \bibinfo{author}{Avila, J.},
  \bibinfo{author}{Frantzeskakis, E.}, \bibinfo{author}{Levy, A.} \&
  \bibinfo{author}{Asensio, M.~C.}
\newblock \bibinfo{title}{Observation of a two-dimensional liquid of
  {Fr\"ohlich} polarons at the bare {SrTiO$_3$} surface}.
\newblock \emph{\bibinfo{journal}{Nat. Commun.}} \textbf{\bibinfo{volume}{6}},
  \bibinfo{pages}{8585} (\bibinfo{year}{2015}).

\bibitem{Cancellieri2016}
\bibinfo{author}{Cancellieri, C.} \emph{et~al.}
\newblock \bibinfo{title}{Polaronic metal state at the {LaAlO$_3$/SrTiO$_3$}
  interface}.
\newblock \emph{\bibinfo{journal}{Nat. Commun}} \textbf{\bibinfo{volume}{7}},
  \bibinfo{pages}{10386} (\bibinfo{year}{2016}).

\bibitem{Baumberger2016}
\bibinfo{author}{Wang, Z.} \emph{et~al.}
\newblock \bibinfo{title}{Tailoring the nature and strength of electron-phonon
  interactions in the {SrTiO$_3$}(001) two-dimensional electron liquid}.
\newblock \emph{\bibinfo{journal}{Nat. Mater.}} \textbf{\bibinfo{volume}{15}},
  \bibinfo{pages}{835--839} (\bibinfo{year}{2016}).

\bibitem{Yukawa2016}
\bibinfo{author}{Yukawa, R.} \emph{et~al.}
\newblock \bibinfo{title}{Phonon-dressed two-dimensional carriers on the {ZnO}
  surface.}
\newblock \emph{\bibinfo{journal}{Phys. Rev. B}} \textbf{\bibinfo{volume}{94}},
  \bibinfo{pages}{165313} (\bibinfo{year}{2016}).

\bibitem{Damascelli2003}
\bibinfo{author}{Damascelli, A.}, \bibinfo{author}{Hussain, Z.} \&
  \bibinfo{author}{Shen, Z.-X.}
\newblock \bibinfo{title}{Angle-resolved photoemission studies of the cuprate
  superconductors}.
\newblock \emph{\bibinfo{journal}{Rev. Mod. Phys.}}
  \textbf{\bibinfo{volume}{75}}, \bibinfo{pages}{473--541}
  (\bibinfo{year}{2003}).

\bibitem{Hardin2012}
\bibinfo{author}{Hardin, B.~E.}, \bibinfo{author}{Snaith, H.~J.} \&
  \bibinfo{author}{McGehee, M.~D.}
\newblock \bibinfo{title}{The renaissance of dye-sensitized solar cells}.
\newblock \emph{\bibinfo{journal}{Nat. Photon.}} \textbf{\bibinfo{volume}{6}},
  \bibinfo{pages}{162--169} (\bibinfo{year}{2012}).

\bibitem{Snaith2014}
\bibinfo{author}{Green, M.~A.}, \bibinfo{author}{Ho-Baillie, A.} \&
  \bibinfo{author}{Snaith, H.~J.}
\newblock \bibinfo{title}{The emergence of perovskite solar cells}.
\newblock \emph{\bibinfo{journal}{Nat. Photon.}} \textbf{\bibinfo{volume}{8}},
  \bibinfo{pages}{506--514} (\bibinfo{year}{2014}).

\bibitem{Fujishima1972}
\bibinfo{author}{Fujishima, A.} \& \bibinfo{author}{Honda, K.}
\newblock \bibinfo{title}{Electrochemical photolysis of water at a
  semiconductor electrode}.
\newblock \emph{\bibinfo{journal}{Nature}} \textbf{\bibinfo{volume}{238}},
  \bibinfo{pages}{37--38} (\bibinfo{year}{1972}).

\bibitem{Fujishima2000}
\bibinfo{author}{Fujishima, A.}, \bibinfo{author}{Rao, T.~N.} \&
  \bibinfo{author}{Tryk, D.~A.}
\newblock \bibinfo{title}{Titanium dioxide photocatalysis}.
\newblock \emph{\bibinfo{journal}{J. Photochem. Photobiol. C}}
  \textbf{\bibinfo{volume}{1}}, \bibinfo{pages}{1--21} (\bibinfo{year}{2000}).

\bibitem{Furubayashi2005}
\bibinfo{author}{Furubayashi, Y.} \emph{et~al.}
\newblock \bibinfo{title}{A transparent metal: Nb-doped anatase {TiO$_2$}}.
\newblock \emph{\bibinfo{journal}{Appl. Phys. Lett.}}
  \textbf{\bibinfo{volume}{86}}, \bibinfo{pages}{252101}
  (\bibinfo{year}{2005}).

\bibitem{Ellmer2012}
\bibinfo{author}{Ellmer, K.}
\newblock \bibinfo{title}{Past achievements and future challenges in the
  development of optically transparent electrodes}.
\newblock \emph{\bibinfo{journal}{Nat. Photon.}} \textbf{\bibinfo{volume}{6}},
  \bibinfo{pages}{809--817} (\bibinfo{year}{2012}).

\bibitem{DeAngelis2014}
\bibinfo{author}{{De Angelis}, F.}, \bibinfo{author}{{Di Valentin}, C.},
  \bibinfo{author}{Fantacci, S.}, \bibinfo{author}{Vittadini, A.} \&
  \bibinfo{author}{Selloni, A.}
\newblock \bibinfo{title}{Theoretical studies on anatase and less common
  {TiO$_2$} phases: Bulk, surfaces, and nanomaterials}.
\newblock \emph{\bibinfo{journal}{Chem. Rev.}} \textbf{\bibinfo{volume}{114}},
  \bibinfo{pages}{9708--9753} (\bibinfo{year}{2014}).

\bibitem{Aryasetiawan1996}
\bibinfo{author}{Aryasetiawan, F.}, \bibinfo{author}{Hedin, L.} \&
  \bibinfo{author}{Karlsson, K.}
\newblock \bibinfo{title}{Multiple plasmon satellites in {Na} and {Al} spectral
  functions from \textit{ab initio} cumulant expansion}.
\newblock \emph{\bibinfo{journal}{Phys. Rev. Lett.}}
  \textbf{\bibinfo{volume}{77}}, \bibinfo{pages}{2268--2271}
  (\bibinfo{year}{1996}).

\bibitem{Guzzo2011}
\bibinfo{author}{Guzzo, M.} \emph{et~al.}
\newblock \bibinfo{title}{Valence electron photoemission spectrum of
  semiconductors: \textit{Ab Initio} description of multiple satellites}.
\newblock \emph{\bibinfo{journal}{Phys. Rev. Lett.}}
  \textbf{\bibinfo{volume}{107}}, \bibinfo{pages}{166401}
  (\bibinfo{year}{2011}).

\bibitem{Lischner2013}
\bibinfo{author}{Lischner, J.}, \bibinfo{author}{Vigil-Fowler, D.} \&
  \bibinfo{author}{Louie, S.~G.}
\newblock \bibinfo{title}{Physical origin of satellites in photoemission of
  doped graphene: An \textit{Ab Initio} {$GW$} plus cumulant study}.
\newblock \emph{\bibinfo{journal}{Phys. Rev. Lett.}}
  \textbf{\bibinfo{volume}{110}}, \bibinfo{pages}{146801}
  (\bibinfo{year}{2013}).

\bibitem{Caruso2015}
\bibinfo{author}{Caruso, F.}, \bibinfo{author}{Lambert, H.} \&
  \bibinfo{author}{Giustino, F.}
\newblock \bibinfo{title}{Band structure of plasmonic polarons}.
\newblock \emph{\bibinfo{journal}{Phys. Rev. Lett.}}
  \textbf{\bibinfo{volume}{114}}, \bibinfo{pages}{146404}
  (\bibinfo{year}{2015}).

\bibitem{Langreth1970}
\bibinfo{author}{Langreth, D.~C.}
\newblock \bibinfo{title}{Singularities in the {X}-ray spectra of metals}.
\newblock \emph{\bibinfo{journal}{Phys. Rev. B}} \textbf{\bibinfo{volume}{1}},
  \bibinfo{pages}{471--477} (\bibinfo{year}{1970}).

\bibitem{Rehr2014}
\bibinfo{author}{Story, S.~M.}, \bibinfo{author}{Kas, J.~J.},
  \bibinfo{author}{Vila, F.~D.}, \bibinfo{author}{Verstraete, M.~J.} \&
  \bibinfo{author}{Rehr, J.~J.}
\newblock \bibinfo{title}{Cumulant expansion for phonon contributions to the
  electron spectral function}.
\newblock \emph{\bibinfo{journal}{Phys. Rev. B}} \textbf{\bibinfo{volume}{90}},
  \bibinfo{pages}{195135} (\bibinfo{year}{2014}).

\bibitem{Hedin1980}
\bibinfo{author}{Hedin, L.}
\newblock \bibinfo{title}{Effects of recoil on shake-up spectra in metals}.
\newblock \emph{\bibinfo{journal}{Phys. Scr.}} \textbf{\bibinfo{volume}{21}},
  \bibinfo{pages}{477--480} (\bibinfo{year}{1980}).

\bibitem{GiustinoRMP}
\bibinfo{author}{Giustino, F.}
\newblock \bibinfo{title}{Electron-phonon interactions from first principles}.
\newblock \emph{\bibinfo{journal}{Rev. Mod. Phys.}}
  \textbf{\bibinfo{volume}{89}}, \bibinfo{pages}{015003}
  (\bibinfo{year}{2017}).

\bibitem{Frohlich1954}
\bibinfo{author}{Fr\"ohlich, H.}
\newblock \bibinfo{title}{Electrons in lattice fields}.
\newblock \emph{\bibinfo{journal}{Adv. Phys.}} \textbf{\bibinfo{volume}{3}},
  \bibinfo{pages}{325--361} (\bibinfo{year}{1954}).

\bibitem{EPW2007}
\bibinfo{author}{Giustino, F.}, \bibinfo{author}{Cohen, M.~L.} \&
  \bibinfo{author}{Louie, S.~G.}
\newblock \bibinfo{title}{Electron-phonon interaction using {W}annier
  functions}.
\newblock \emph{\bibinfo{journal}{Phys. Rev. B}} \textbf{\bibinfo{volume}{76}},
  \bibinfo{pages}{165108} (\bibinfo{year}{2007}).

\bibitem{Verdi2015}
\bibinfo{author}{Verdi, C.} \& \bibinfo{author}{Giustino, F.}
\newblock \bibinfo{title}{Fr\"ohlich electron-phonon vertex from first
  principles}.
\newblock \emph{\bibinfo{journal}{Phys. Rev. Lett.}}
  \textbf{\bibinfo{volume}{115}}, \bibinfo{pages}{176401}
  (\bibinfo{year}{2015}).

\bibitem{EPW2016}
\bibinfo{author}{Ponc\'e, S.}, \bibinfo{author}{Margine, R.},
  \bibinfo{author}{Verdi, C.} \& \bibinfo{author}{Giustino, F.}
\newblock \bibinfo{title}{{EPW}: Electron-phonon coupling, transport and
  superconducting properties using maximally localized {W}annier functions}.
\newblock \emph{\bibinfo{journal}{Comput. Phys. Commun.}}
  \textbf{\bibinfo{volume}{209}}, \bibinfo{pages}{116--133}
  (\bibinfo{year}{2016}).

\bibitem{Grimvall}
\bibinfo{author}{Grimvall, G.}
\newblock \emph{\bibinfo{title}{The Electron-Phonon Interaction in Metals}}
  (\bibinfo{publisher}{North-Holland}, \bibinfo{address}{New York},
  \bibinfo{year}{1981}).

\bibitem{Moser2015}
\bibinfo{author}{Moser, S.} \emph{et~al.}
\newblock \bibinfo{title}{Electron-phonon coupling in the bulk of anatase
  {TiO$_2$} measured by resonant inelastic {X}-ray spectroscopy}.
\newblock \emph{\bibinfo{journal}{Phys. Rev. Lett.}}
  \textbf{\bibinfo{volume}{115}}, \bibinfo{pages}{096404}
  (\bibinfo{year}{2015}).

\bibitem{Marsiglio1990}
\bibinfo{author}{Marsiglio, F.}
\newblock \bibinfo{title}{Pairing and charge-density-wave correlations in the
  {H}olstein model at half-filling}.
\newblock \emph{\bibinfo{journal}{Phys. Rev. B}} \textbf{\bibinfo{volume}{42}},
  \bibinfo{pages}{2416--2424} (\bibinfo{year}{1990}).

\bibitem{Emin}
\bibinfo{author}{Emin, D.}
\newblock \emph{\bibinfo{title}{Polarons}} (\bibinfo{publisher}{Cambridge
  University Press}, \bibinfo{year}{2012}).

\bibitem{Setvin2014}
\bibinfo{author}{Setvin, M.} \emph{et~al.}
\newblock \bibinfo{title}{Direct view at excess electrons in {TiO$_2$} rutile
  and anatase}.
\newblock \emph{\bibinfo{journal}{Phys. Rev. Lett.}}
  \textbf{\bibinfo{volume}{113}}, \bibinfo{pages}{086402}
  (\bibinfo{year}{2014}).

\bibitem{Gervais1993}
\bibinfo{author}{Gervais, F.}, \bibinfo{author}{Servoin, J.-L.},
  \bibinfo{author}{Baratoff, A.}, \bibinfo{author}{Bednorz, J.~G.} \&
  \bibinfo{author}{Binnig, G.}
\newblock \bibinfo{title}{Temperature dependence of plasmons in {Nb}-doped
  {SrTiO$_3$}}.
\newblock \emph{\bibinfo{journal}{Phys. Rev. B}} \textbf{\bibinfo{volume}{47}},
  \bibinfo{pages}{8187--8194} (\bibinfo{year}{1993}).

\bibitem{Goldstein1982}
\bibinfo{author}{Many, A.}, \bibinfo{author}{Gersten, J.~I.},
  \bibinfo{author}{Wagner, I.}, \bibinfo{author}{Rosenthal, A.} \&
  \bibinfo{author}{Goldstein, Y.}
\newblock \bibinfo{title}{Two-dimensional plasma oscillation on {ZnO}
  surfaces}.
\newblock \emph{\bibinfo{journal}{Surf. Sci.}} \textbf{\bibinfo{volume}{113}},
  \bibinfo{pages}{355--361} (\bibinfo{year}{1982}).

\bibitem{Horn1972}
\bibinfo{author}{Horn, M.}, \bibinfo{author}{Schwerdtfeger, C.~F.} \&
  \bibinfo{author}{Meagher, E.~P.}
\newblock \bibinfo{title}{Refinement of the structure of anatase at several
  temperatures}.
\newblock \emph{\bibinfo{journal}{Z. Kristallogr.}}
  \textbf{\bibinfo{volume}{136}}, \bibinfo{pages}{273--281}
  (\bibinfo{year}{1972}).

\bibitem{PBE1996}
\bibinfo{author}{Perdew, J.~P.}, \bibinfo{author}{Burke, K.} \&
  \bibinfo{author}{Ernzerhof, M.}
\newblock \bibinfo{title}{Generalized gradient approximation made simple}.
\newblock \emph{\bibinfo{journal}{Phys. Rev. Lett.}}
  \textbf{\bibinfo{volume}{77}}, \bibinfo{pages}{3865--3868}
  (\bibinfo{year}{1996}).

\bibitem{QuantumEspresso}
\bibinfo{author}{Giannozzi, P.} \emph{et~al.}
\newblock \bibinfo{title}{{QUANTUM ESPRESSO}: a modular and open-source
  software project for quantum simulations of materials}.
\newblock \emph{\bibinfo{journal}{J. Phys. Condens. Matter}}
  \textbf{\bibinfo{volume}{21}}, \bibinfo{pages}{395502}
  (\bibinfo{year}{2009}).

\bibitem{Zhukov2014}
\bibinfo{author}{Zhukov, V.~P.} \& \bibinfo{author}{Chulkov, E.~V.}
\newblock \bibinfo{title}{Ab initio calculations of the electron-phonon
  interaction and characteristics of large polarons in rutile and anatase}.
\newblock \emph{\bibinfo{journal}{Phys. Solid State}}
  \textbf{\bibinfo{volume}{56}}, \bibinfo{pages}{1302--1309}
  (\bibinfo{year}{2014}).

\bibitem{Mott1990}
\bibinfo{author}{Mott, N.~F.}
\newblock \emph{\bibinfo{title}{Metal-insulator transitions}}
  (\bibinfo{publisher}{Taylor\&Francis}, \bibinfo{address}{London},
  \bibinfo{year}{1990}).

\bibitem{Hedin1965}
\bibinfo{author}{Hedin, L.}
\newblock \bibinfo{title}{New method for calculating the one-particle {G}reen's
  function with application to the electron-gas problem}.
\newblock \emph{\bibinfo{journal}{Phys. Rev.}} \textbf{\bibinfo{volume}{139}},
  \bibinfo{pages}{A796--A823} (\bibinfo{year}{1965}).

\bibitem{Mardare2004}
\bibinfo{author}{Mardare, D.} \& \bibinfo{author}{Rusu, G.~I.}
\newblock \bibinfo{title}{Comparison of the dielectric properties for doped and
  undoped {TiO$_2$}}.
\newblock \emph{\bibinfo{journal}{J. Optoelectron. Adv. Mater.}}
  \textbf{\bibinfo{volume}{6}}, \bibinfo{pages}{333--336}
  (\bibinfo{year}{2004}).

\bibitem{ARC}
\bibinfo{author}{Richards, A.}
\newblock \bibinfo{title}{University of {O}xford advanced research computing.
  http://dx.doi.org/10.5281/zenodo.22558}  (\bibinfo{year}{2015}).

\end{thebibliography}

\clearpage

\section*{\textsf{Acknowledgements}} \vspace*{-10pt}
The authors gratefully acknowledge Marco~Grioni for granting permission to reproduce Fig.~1a-c. 
This work was supported by the Leverhulme Trust (Grant RL-2012-001), the Graphene Flagship 
(Horizon 2020 Grant No. 696656 -- GrapheneCore1), and the UK Engineering and Physical 
Sciences Research Council (Grant No. EP/M020517/1). This work used the ARCHER UK 
National Supercomputing Service via the ’AMSEC’ Leadership project, and the Advanced Research 
Computing facility of the University of Oxford~\cite{ARC}. 

\section*{\textsf{Author contributions}} \vspace*{-10pt}
C.V. performed the computational research and analysed the results. 
F.C. performed the cumulant expansion. 
F.G. designed the research and led the project. 
All authors participated in the preparation of the manuscript. 

\section*{\textsf{Additional information}} \vspace*{-10pt}
\begin{addendum}
\item[Supplementary Information]  
Supplementary information accompanies this paper at [URL]. \vspace*{-14pt}
\item[Competing financial interests] The authors declare no competing financial interests.
\end{addendum} 

\clearpage

\small
  \begin{figure}
  \caption{\label{fig1}
  \textbf{$\!\!\bm|\,$ \textit{Ab initio} ARPES spectra of \textit{n}-doped anatase TiO$_{\textbf{2}}$.}
  \normalfont(\textbf{a}-\textbf{c}) ARPES spectra of anatase TiO$_2$ measured by Moser \textit{et al.}\cite{Moser2013}.
  The measurements were taken at 20~K on samples with $5\times10^{18}$~cm$^{-3}$ (\textbf{a}),
  $3\times10^{19}$~cm$^{-3}$ (\textbf{b}), and $3.5\times10^{20}$~cm$^{-3}$ (\textbf{c}). The
  zero of the energy is set to the Fermi level. The electron momentum $k_x$ is along the 
  $\Gamma\Sigma$ line of the anatase Brillouin zone (see panel {\bf j}). Reproduced with permission from 
  ref.~\citenum{Moser2013}. Copyright 2013 American Physical Society. 
  (\textbf{d}-\textbf{f}) Calculated spectral function of anatase TiO$_2$, for the same electron momenta 
  and nominal doping levels as in \textbf{a}-\textbf{c}.
  Gaussian masks of widths 25~meV and 0.015~\AA$^{-1}$ were applied to account for the experimental
  resolution\cite{Moser2013}. (\textbf{g}-\textbf{i}) Band structures extracted from the calculated 
  spectral functions in \textbf{d}-\textbf{f}. The bare bands are in red, the bands including electron-phonon 
  interactions are in blue. The calculated mass enhancement parameter $\lambda$ is 0.73 (\textbf{g}), 
  0.70 (\textbf{h}) and 0.20 (\textbf{i}). 
  (\textbf{j}) Brillouin zone and high-symmetry lines of anatase TiO$_2$.
  (\textbf{k}) Calculated ARPES spectrum for a doping concentration of $3\times10^{19}$~cm$^{-3}$, 
  showing the anisotropy of the electron dispersions along $\Gamma$X (basal plane of the tetragonal 
  lattice, see Supplementary Fig.~1) and $\Gamma$Z ($c$-axis). }
  \end{figure}

\clearpage

 \begin{figure}
 \caption{\label{fig2}
 \textbf{$\!\!\bm|\,$ Origin of satellites and kinks, and polaron wavefunctions.}
 \normalfont(\textbf{a}) Effect of high-energy phonons: we compare the spectral function calculated for 
 the intermediate doping level ($3\times10^{19}$~cm$^{-3}$) by taking into account all vibrational modes 
 and a calculation where all phonons with energy above 100~meV have been eliminated. 
 (\textbf{b}) Effect of electron-phonon correlations beyond the one-shot Migdal approximation: 
 the complete calculation using the cumulant expansion method is compared to a spectral 
 function calculated within the one-shot Migdal approximation. The doping level is the same as in \textbf{a}.
 (\textbf{c}) Effect of dynamical screening on the spectral function: 
 we compare the spectral function calculated for the highest doping level
 ($3.5\times10^{20}$~cm$^{-3}$) by taking into account the screening of the electron-phonon 
 interaction by carriers, with a calculation where this effect is turned off, and as a result the 
 electron-phonon coupling is artificially enhanced. For clarity these spectra were not convoluted 
 with Gaussian masks as in Fig.~\ref{fig1}d-f. (\textbf{d}) Identification of polaronic 
 region and Fermi liquid region in $n$-doped anatase TiO$_2$: the red spheres represent the plasmon 
 energy at each doping level, the horizontal line is the energy of the LO $E_u$ phonon (109~meV). 
 The electron-phonon coupling strength $\lambda$ is given by the blue spheres. The lines are guides 
 to the eye. (\textbf{e}) Ball-and-stick representation of the LO $E_u$ phonon, showing for clarity only 
 one of the TiO$_6$ octahedra. 
 (\textbf{f}) Square moduli of the polaron wavefunctions near the origin and further away from the 
 origin, in the polaronic region (blue, $3\times10^{19}$~cm$^{-3}$) and in the Fermi liquid region 
 (yellow, $3.5\times10^{20}$~cm$^{-3}$). The corresponding envelope functions are shown 
 as the blue and red curves, respectively. These wavefunctions are extended over all three Cartesian 
 directions, but are only shown along the $c$-axis for clarity. }
 \end{figure}


\end{document}
